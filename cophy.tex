\documentclass{article}
\usepackage{amsmath}
\usepackage{amssymb}
\frenchspacing

\makeatletter
\newcommand{\vast}{\bBigg@{4}}
\newcommand{\Vast}{\bBigg@{5}}
\makeatother

\title{Bayesian Inference of Cophylogenies}

\begin{document}

    \section*{Methods}

        \subsection*{The Model}

            The host and symbiont processes originate together. The host
            evolves under a pure birth process and simultaneous with every host
            speciation event is the cospeciation of every symbiont lineage
            currently associated with that host. When a symbiont cospeciates,
            one of the new lineages is paired with one of the new host lineages
            and the other symbiont lineage is paired with the other host
            lineage. Independently of the evolution of its host, a symbiont may
            speciate with both new lineages remaining on the host (duplication
            event), speciate with one of the new lineages remaining on the host
            and the other symbiont lineage selecting a uniformly random new
            host (host-switch event), or go extinct (loss event). These events
            are independent Poisson processes with rates $\lambda$, $\tau$, and
            $\mu$, respectively. No host-switching may occur when there is only
            a single host lineage.

        \subsection*{Definitions}

            Let $S$ and $H$ be the set of nodes in the symbiont and host trees,
            respectively. Both are rooted, bifurcating, ultrametric
            reconstructed trees that include an additional origin node that is
            ancestral to their root nodes. We assume that the height of all
            leaf nodes is $t = 0$.

            Let $t_n$ be the height of a node~$n$ and $H_t = \left\{h \in H:
            t_h \leq t < t_p\right\}$ be the set of host lineages that exist
            at time~$t$, where $p$ is the parent of $h$.

        \subsection*{Density for the Reconstructed Coevolutionary Process}

            Let $\vec{S}\left(t\right) : \mathbb{R}_{\geq 0} \to
            \mathbb{N}_{0}^{\lvert{H_t}\rvert}$ be the number of symbionts
            present on each host lineage at time~$t$. Assuming that in an
            infinitesimal period of time there may be only one duplication,
            host-switch, or loss event, we have the master equation

            \begin{equation}
                \begin{split}
                    \frac{\partial}{\partial t} P\left(\vec{s},t\right) =
                    \sum_{i=1}^{\lvert{H_t}\rvert}
                    &P\left(\vec{s} - \hat{u}_i, t\right)T_{B_i}\left(\vec{s} -
                    \hat{u}_i\right) \\ &+ P\left(\vec{s} + \hat{u}_i,
                    t\right)T_{D_i}\left(\vec{s} + \hat{u}_i \right) \\ &-
                    P\left(\vec{s}, t\right)\left(T_{B_i}\left(\vec{s}\right) +
                    T_{D_i}\left(\vec{s} \right)\right)
                \end{split}
            \end{equation}
            with the condition that $P\left(\vec{s},t\right) = 0$ if any
            component of $\vec{s}$ is less than~$0$, where $P\left(\vec{s},
            t\right) \equiv P\left(\vec{S}\left(t\right)
            = \vec{s}, t\right)$, $\hat{u}_i$ is the unit vector whose $i$th
            component is~$1$,
            \begin{equation}
                T_{B_i}\left(\vec{s}\right) =
                \begin{cases}
                    \lambda \vec{s}_i & \lvert{H_t}\rvert = 1 \\
                    \lambda \vec{s}_i +
                    \frac{\tau}{\lvert{H_t}\rvert - 1}
                    \left(\sum_{j=1}^{\lvert{H_t}\rvert}
                    {\vec{s}_j}-\vec{s}_i\right) & \lvert{H_t}\rvert > 1 \\
                \end{cases},
            \end{equation}
            and
            \begin{equation}
                T_{D_i}\left(\vec{s}\right) = \mu \vec{s}_i.
            \end{equation}
            Note that $P\left(\vec{s},t\right)$ is differentiable only where
            $\lvert{H_t}\rvert$ is continuous.

            If we add the condition that $P\left(\vec{s},t\right) = 0$ if any
            component of $\vec{s}$ is more than some limit~$L$ and that
            $T_{B_i}\left(\vec{s}\right) = 0$ if $\vec{s}_i = L$ (i.e., we
            assume that the system may transition between only a finite set of
            states), we may write the master equation as
            \begin{equation}
                \frac{\partial}{\partial t} \vec{P}\left(t\right) =
                Q\vec{P}\left(t\right),
            \end{equation}
            where $\vec{P}\left(t\right)$ is a vector consisting of
            $P\left(\vec{s}, t\right)$ for all values of $\vec{s}$ for which
            $P\left(\vec{s}, t\right)$ may be greater than 0 and $Q$ is a
            sparse $(L+1)^{\lvert{H_t}\rvert} \times (L+1)^{\lvert{H_t}\rvert}$
            matrix consisting of the coefficients from equation (1). The
            solution to (4) for intervals of time where the number of hosts is
            constant is
            \begin{equation}
                \vec{P}\left(t\mid\vec{p}_i, t_i\right) =
                e^{Q\left(t-t_i\right)}\vec{p}_i,
            \end{equation}
            where $\vec{p}_i$ is the initial density at some time $t_i$. We may
            extend this solution to intervals of time in which the number of
            hosts increases (i.e., there are cospeciation events) by
            introducing a function $f$ that maps the density immediately
            preceding a cospeciation to that immediately following the
            cospeciation event.

            The remaining problem is to accomodate speciation events in the
            symbiont phylogeny. Let $P_\varnothing\left(\vec{p}_i, t_i\right)$
            be the probability that there are no symbiont lineages on any host
            at time $t = 0$ given the state density $\vec{p}_i$ at time $t_i$;
            i.e., the component of $\vec{P}\left(0\mid\vec{p}_i, t_i\right)$
            associated with the state vector $\langle0,\dots,0\rangle$.  If $n
            \in S$ is a leaf node and $h \in H$ its host, let
            $P_n\left(\vec{p}_i, t_i\right)$ be the component of
            $\vec{P}\left(0\mid\vec{p}_i, t_i\right)$ associated with the state
            vector $\hat{u}_h$, the unit vector whose component associated with
            $h$ is 1. Otherwise, we distinguish between cospeciation events and
            independent speciation events:
            \begin{equation}
                P_n\left(\vec{p}_i, t_i\right) =
                \begin{cases}
                    \begin{aligned}
                        \left(P_{c_1}\left(\hat{u}_{k_1}, t_n\right)
                        P_{c_2}\left(\hat{u}_{k_2}, t_n\right) +
                        P_{c_1}\left(\hat{u}_{k_2}, t_n\right)
                        P_{c_2}\left(\hat{u}_{k_1}, t_n\right)\right) \\
                        \cdot P_\varnothing\left(f\left(g_h\left(\vec{P}\left(t_n\mid
                        \vec{p}_i, t_i\right)\right)\right), t_n\right)
                    \end{aligned} & t_n = t_h \\[8ex]
                    \begin{aligned}
                        & \vast( 2\lambda P_{c_1}\left(\hat{u}_h,
                        t_n\right) P_{c_2}\left(\hat{u}_h, t_n\right) +
                        \frac{\tau}{\lvert{H_t}\rvert-1} \\
                        &\cdot\left(P_{c_1}\left(\hat{u}_h, t_n\right)
                        \sum_{j\in H_t,j\neq h}{P_{c_2}\left(\hat{u}_j,
                        t_n\right)} + P_{c_2}\left(\hat{u}_h, t_n\right)
                        \sum_{j\in H_t,j\neq h}
                        {P_{c_1}\left(\hat{u}_j,t_n\right)}\right)\vast) \\
                        & \cdot P_\varnothing\left(g_h\left(\vec{P}\left(t_n\mid\vec{p}_i,
                        t_i\right)\right), t_n\right)
                    \end{aligned}
                    & t_n \neq t_h
                \end{cases},
            \end{equation}
            where $c_1$ and $c_2$ are the children of $n$, $k_1$ and $k_2$
            are the children of $h$, and $g_h$ maps a state density to the
            density where one symbiont lineage has been removed from the host
            lineage $h$.

            The initial density $\vec{p}_{or}$ for the reconstructed process is
            $P\left(\langle1\rangle, t_\text{or}\right) = 1$ and 0 for all
            other $\vec{s}$ at time~$t_\text{or}$, such that the density of the
            reconstructed cophylogeny is $P_r\left(\vec{p}_{or},
            t_{or}\right)$, where $r$ is the root of the symbiont phylogeny.


        \subsection*{Operators}

            \subsubsection*{Host-Switch Operator}

                Randomly select an internal node $i$ from the symbiont tree.
                Let $p$ be its parent, $c_1$ and $c_2$ be its children, $h \in
                H_{t_i}$ be its host, $q$ be the parent of $h$, $t_\text{lower}
                = \max\left(t_{c_1},t_{c_2}\right)$, and $t_\text{upper} =
                \min\left(t_p,t_q\right)$. If $t_i = t_h$ then no operation is
                possible.\footnote{This seems like a reasonable condition, as
                we should not be able to perform this operation when we are
                missing the dimension that we are operating in (i.e., in the
                cospeciating state). However, in practice, it heavily skews the
                MCMC to spend substantial time in the cospeciating state.}
                Otherwise, choose a new height ${t_i}^*$ uniformly at random
                from the interval $\left[t_\text{lower}, t_\text{upper}\right]$
                and then randomly select a new host $h^* \in
                H_{{t_i}^*}.$\footnote{This is not the operator exactly as
                Alexei suggested it, but this version is easier to implement.}
                The Hastings ratio for this operation is
                $\frac{\lvert{H_{{t_i}^*}}\rvert} {\lvert{H_{t_i}}\rvert}$.

            \subsubsection*{Cospeciation Operator}

                Randomly select an internal node $i$ from the symbiont tree.
                Let $p$ be its parent, $c_1$ and $c_2$ be its children, $h \in
                H_{t_i}$ be its host, $q$ be the parent of $h$, and
                $t_\text{upper} = \min\left(t_p,t_q\right)$. No operation is
                possible when $t_h \leq \max\left(t_{c_1},t_{c_2}\right)$.

                If $t_i = t_h$, choose a new height $t_i^*$ uniformly at random
                from the interval $\left[t_h, t_\text{upper}\right]$. This
                operation increases the state's dimension by one and has a
                Hastings ratio of $t_\text{upper} - t_h$.

                Otherwise, when $t_i \neq t_h$, set $t_i^* = t_h$. This reduces
                the dimension by one and has a Hastings ratio of
                $\frac{1}{t_\text{upper} - t_h}$.
\end{document}
