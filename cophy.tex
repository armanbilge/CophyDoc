\documentclass{article}
\usepackage{amsmath}
\usepackage{amssymb}
\frenchspacing

\title{Bayesian Inference of Cophylogenies}

\begin{document}

    \section*{Methods}

        \subsection*{The Model}

            The host and symbiont processes originate together. The host
            evolves under a pure birth process and simultaneous with every host
            speciation event is the cospeciation of every symbiont lineage
            currently associated with that host. When a symbiont cospeciates,
            one of the new lineages is paired with one of the new host lineages
            and the other symbiont lineage is paired with the other host
            lineage. Independently of the evolution of its host, a symbiont may
            speciate with both new lineages remaining on the host (duplication
            event), speciate with one of the new lineages remaining on the host
            and the other symbiont lineage selecting a uniformly random new
            host (host-switch event), or go extinct (loss event). These events
            are independent Poisson processes with rates $\lambda$, $\tau$, and
            $\mu$, respectively. No host-switching may occur when there is only
            a single host lineage.

        \subsection*{Definitions}

            Let , and . The complete cophylogeny consists

            Let $\tau_n$ be the height of a node~$n$ and $H_t = \left\{h \in H:
            \tau_h \leq t < \tau_p\right\}$ be the set of host lineages that exist
            at time~$t$, where $p$ is the parent of $h$.

        \subsection*{Cophylogeny Density}

            We partition the cophylogeny into intervals of time between
            symbiont speciation events.

            If $s$ is a leaf node, we return the probability of a single
            symbiont on $h = \text{host}\left(s\right)$, $P\left(\hat{u}_h,
            0\right)$, where $\hat{u}_i$ is the unit vector whose $i$th
            component is~$1$.

            If $s$ is a cospeciation event then we return

        \subsection*{Density for the D-HS-L Process}

            Let $\vec{S}\left(t\right) : \mathbb{R}_{\geq 0} \to
            \mathbb{N}_{0}^{\lvert{H_t}\rvert}$ be the number of symbionts
            present on each host lineage at time~$t$. Assuming that in an
            infinitesimal period of time there may be only one duplication,
            host-switch, or loss event,

            \begin{equation}
                \begin{split}
                    \frac{\partial}{\partial t} P\left(\vec{s},t\right) =
                    \sum_{i=1}^{\lvert{H_t}\rvert}
                    &P\left(\vec{s} - \hat{u}_i, t\right)T_{B_i}\left(\vec{s} -
                    \hat{u}_i\right) \\ &+ P\left(\vec{s} + \hat{u}_i,
                    t\right)T_{D_i}\left(\vec{s} + \hat{u}_i \right) \\ &-
                    P\left(\vec{s}, t\right)\left(T_{B_i}\left(\vec{s}\right) +
                    T_{D_i}\left(\vec{s} \right)\right)
                \end{split}
            \end{equation}
            with the condition that $P\left(\vec{s},t\right) = 0$ if any
            component of $\vec{s}$ is less than~$0$, where $P\left(\vec{s},
            t\right) \equiv P\left(\vec{S}\left(t\right)
            = \vec{s}, t\right)$, $\hat{u}_i$ is the unit vector whose $i$th
            component is~$1$,
            \begin{equation}
                T_{B_i}\left(\vec{s}\right) =
                \begin{cases}
                    \lambda \vec{s}_i & \lvert{H_t}\rvert = 1 \\
                    \lambda \vec{s}_i +
                    \frac{\tau}{\lvert{H_t}\rvert - 1}
                    \left(\sum_{j=1}^{\lvert{H_t}\rvert}
                    {\vec{s}_j}-\vec{s}_i\right) & \lvert{H_t}\rvert > 1 \\
                \end{cases}
            \end{equation}
            and
            \begin{equation}
                T_{D_i}\left(\vec{s}\right) = \mu \vec{s}_i.
            \end{equation}

            If we add the condition that $P\left(\vec{s},t\right) = 0$ if any
            component of $\vec{s}$ is more than some maximum~$L$ and that
            $T_{B_i}\left(\vec{s}\right) = 0$ if $\vec{s}_i = L$ (i.e., we
            assume that the system may transition between only a finite set of
            states), we may write the master equation as
            \begin{equation}
                \frac{\partial}{\partial t} \vec{P}\left(t\right) =
                Q\vec{P}\left(t\right)
            \end{equation}
            where $\vec{P}\left(t\right)$ is a vector consisting of
            $P\left(\vec{s}, t\right)$ for all possible values of $\vec{s}$ and
            $Q$ is a matrix consisting of the coefficients from equation (1).
            The solution to (4) is
            \begin{equation}
                \vec{P}\left(t\right) = e^{Qt}\vec{p}_0
            \end{equation}
            where $\vec{p}_0$ is the initial density for the system.

        \subsection*{Density for the Complete Coevolutionary Process}

            The initial density for the next interval can be calculated using
            the final density of the previous interval using a function
            $f : [0,1]^n \to [0,1]^{n+1}$ that maps the density immediately
            preceding the cospeciation event to that immediately following the
            cospeciation event.

            The initial density for the entire process is
            $P\left(\langle1\rangle, t_\text{or}\right) = 1$ and 0 for all
            other $\vec{s}$ at time~$t_\text{or}$.

        \subsection*{Operators}

            \subsubsection*{Host-Switch Operator}

                Randomly select an internal node $i$ from the symbiont tree.
                Let $p$ be its parent, $c_1$ and $c_2$ be its children, $h \in
                H_{\tau_i}$ be its host, $q$ be the parent of $h$,
                $\tau_\text{lower} = \max\left(\tau_{c_1},\tau_{c_2}\right)$,
                and $\tau_\text{upper} = \min\left(\tau_p,\tau_q\right)$.
                Choose a new height ${\tau_i}^*$ uniformly at random from
                the interval $\left[\tau_\text{lower},
                \tau_\text{upper}\right]$ and then randomly select a new host
                $h^* \in H_{{\tau_i}^*}.$\footnote{This is not the operator
                exactly as Alexei suggested it, but this version is easier to
                implement.} The Hastings ratio for this operation is
                $\frac{\lvert{H_{{\tau_i}^*}}\rvert}
                {\lvert{H_{\tau_i}}\rvert}$.

            \subsubsection*{Cospeciation Operator}

                Randomly select an internal node $i$ from the symbiont tree.
                Let $p$ be its parent, $c_1$ and $c_2$ be its children, $h \in
                H_{\tau_i}$ be its host, $q$ be the parent of $h$, and
                $\tau_\text{upper} = \min\left(\tau_p,\tau_q\right)$. No
                operation is possible when $\tau_h \leq
                \max\left(\tau_{c_1},\tau_{c_2}\right)$.

                If $\tau_i = \tau_h$, choose a new height $\tau_i^*$ uniformly
                at random from the interval $\left[\tau_h,
                \tau_\text{upper}\right]$. This operation increases the state's
                dimension by one and has a Hastings ratio of $\tau_\text{upper}
                - \tau_h$.

                Otherwise when $\tau_i \neq \tau_h$, set $\tau_i^* = \tau_h$.
                This reduces the dimension by one and has a Hastings ratio of
                $\frac{1}{\tau_\text{upper} - \tau_h}$.
\end{document}
